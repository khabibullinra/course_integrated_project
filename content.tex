\begin{frame}
\titlepage
\end{frame}

% слайд с инофрмацией об авторах
% кратко написать о себе, оставить контанты
% может быть надо встравить фотки какие то
\begin{frame}{Информация об авторах курса}
\begin{itemize}
    \item Кондаков Данила Евгеньевич
    \item Хабибуллин Ринат Альфредович (khabibullin.ra@gubkin.ru)
\end{itemize}
    
\end{frame}

% слайд с описанием 
% чтобы и самим не запутаться - что хотим донести то
\begin{frame}{Описание курса}
\begin{itemize}
    \item Цели курса - получить представление об основных ключевых факторах влияющих на эффективность разработки месторождений нефти, изучить цепочку создания ценности при разработке месторождений.
    \item Инструмент изучения - разбор задачи интегрированного концептуального проектирования разработки месторождений.
    \item Задачи - получить навыки проведения инженерных расчетов ключевых факторов разработки с использованием компьютерных моделей.
    \item Курс носит практическую направленность - лучше все изучать разработку через расчеты!
\end{itemize}
\end{frame}

\begin{frame}{Инраструктура курса}
\begin{itemize}
    \item сайт поддержки курса \href{rienm.ru}{rienm.ru}
    \item курс "ИРНП-2020", "Основы интегрированной разработки нефтяных проектов"
    \item на сайте - дополнительные материалы, инструкции, задания и пр.
    \item надо зарегистрироваться и подписаться на курс. ключевое слово для записи на курс "ISKRA"
\end{itemize}
    
\end{frame}

% слайд содержание курса
%   - основные разделы по которым будет построен курс
\begin{frame}{Содержание курса}
\begin{enumerate}
    \item Введение. Основы разработки нефтяных месторождений.
    \item Заканчивание, кустование и бурение скважин.
    \item Выбор скважинного оборудования и поверхностного обустройства.
    \item Финансово-экономическая модель.
    \item Оптимизация интегрированного концептуального проекта. 
    \item Работа над проектным заданием.
\end{enumerate}
\end{frame}

% Блок вводных слайдов, которые описывают задачу интегрированного концептуального проектирования
%
\begin{frame}{Жизненный цикл месторождения. Проект разработки}

Тут надо будет показать картинки жизненного цикла от разведки через разработку и ликвидацию месторождения. 
На картинке показать где актуальная задача проектирования. 

\end{frame}

\begin{frame}{Моделирование в разработке нефтяных проектов}
Какова роль моделирования в нефтяных проектах. Цели и задачи моделирования. Проектирование и моделирование. Управление месторождением и моделирование. Моделирование и принятие решений. 
\end{frame}

\begin{frame}{Интегрированный проект}

Интегрированный проект - такой где совместно решаются разномасштабные задачи разработки - фильтрация в пласте, работа скважины, скважинного оборудования и поверхностного обустройства.

Надо тут пояснить сложность интегрированной задачи - большое количество переменных на каждом этапе и необходиость "сшивки" решений.
    
\end{frame}

\begin{frame}{Интегрированный концептуальный проект}

Интегрированный концептуальный проект - один из подходов к созданию интегрированных проектов. Сосредотачивается на ключевых решениях разработки - выбор сетки скважины, типа заканчивания скважины, способа подъема жидкости на поверхность, основных узлов поверхностного обустройства.

Подход снижает количество переменных при оптимизации. Но не больше чем нужно!
    
\end{frame}

\begin{frame}{Расчет интегрированного концептуального проекта}

Интегрированный концептуальный проект - один из подходов к созданию интегрированных проектов. Сосредотачивается на ключевых решениях разработки - выбор сетки скважины, типа заканчивания скважины, способа подъема жидкости на поверхность, основных узлов поверхностного обустройства.

Подход снижает количество переменных при оптимизации. Но не больше чем нужно!
    
\end{frame}

\begin{frame}{Перечень использованных источников}
\printbibliography[title=Этот цвет тоже можно поменять.]
\end{frame}


\begin{frame}{test}

% diagram prepared in https://www.mathcha.io/


\tikzset{every picture/.style={line width=0.75pt}} %set default line width to 0.75pt        

\begin{tikzpicture}[x=0.75pt,y=0.75pt,yscale=0.5,xscale=0.5]
%uncomment if require: \path (0,487); %set diagram left start at 0, and has height of 487

%Rounded Rect [id:dp7252804105471826] 
\draw  [fill={rgb, 255:red, 182; green, 244; blue, 115 }  ,fill opacity=1 ] (193,13) .. controls (193,8.58) and (196.58,5) .. (201,5) -- (255,5) .. controls (259.42,5) and (263,8.58) .. (263,13) -- (263,37) .. controls (263,41.42) and (259.42,45) .. (255,45) -- (201,45) .. controls (196.58,45) and (193,41.42) .. (193,37) -- cycle ;

%Rounded Rect [id:dp19699088328362624] 
\draw  [fill={rgb, 255:red, 182; green, 244; blue, 115 }  ,fill opacity=1 ] (342,14) .. controls (342,9.58) and (345.58,6) .. (350,6) -- (486.5,6) .. controls (490.92,6) and (494.5,9.58) .. (494.5,14) -- (494.5,38) .. controls (494.5,42.42) and (490.92,46) .. (486.5,46) -- (350,46) .. controls (345.58,46) and (342,42.42) .. (342,38) -- cycle ;

%Rounded Rect [id:dp05428373198201508] 
\draw  [fill={rgb, 255:red, 197; green, 220; blue, 245 }  ,fill opacity=1 ] (149,177.07) .. controls (149,171.51) and (153.51,167) .. (159.07,167) -- (295.43,167) .. controls (300.99,167) and (305.5,171.51) .. (305.5,177.07) -- (305.5,207.27) .. controls (305.5,212.83) and (300.99,217.33) .. (295.43,217.33) -- (159.07,217.33) .. controls (153.51,217.33) and (149,212.83) .. (149,207.27) -- cycle ;

%Rounded Rect [id:dp43171100329887246] 
\draw  [fill={rgb, 255:red, 197; green, 220; blue, 245 }  ,fill opacity=1 ] (150,93.07) .. controls (150,87.51) and (154.51,83) .. (160.07,83) -- (296.43,83) .. controls (301.99,83) and (306.5,87.51) .. (306.5,93.07) -- (306.5,123.27) .. controls (306.5,128.83) and (301.99,133.33) .. (296.43,133.33) -- (160.07,133.33) .. controls (154.51,133.33) and (150,128.83) .. (150,123.27) -- cycle ;

%Rounded Rect [id:dp7487500578781445] 
\draw  [fill={rgb, 255:red, 249; green, 226; blue, 226 }  ,fill opacity=1 ] (147,263.07) .. controls (147,257.51) and (151.51,253) .. (157.07,253) -- (293.43,253) .. controls (298.99,253) and (303.5,257.51) .. (303.5,263.07) -- (303.5,293.27) .. controls (303.5,298.83) and (298.99,303.33) .. (293.43,303.33) -- (157.07,303.33) .. controls (151.51,303.33) and (147,298.83) .. (147,293.27) -- cycle ;

%Rounded Rect [id:dp3136002014262538] 
\draw  [fill={rgb, 255:red, 249; green, 226; blue, 226 }  ,fill opacity=1 ] (222,433.07) .. controls (222,427.51) and (226.51,423) .. (232.07,423) -- (368.43,423) .. controls (373.99,423) and (378.5,427.51) .. (378.5,433.07) -- (378.5,463.27) .. controls (378.5,468.83) and (373.99,473.33) .. (368.43,473.33) -- (232.07,473.33) .. controls (226.51,473.33) and (222,468.83) .. (222,463.27) -- cycle ;

%Rounded Rect [id:dp48591184210337124] 
\draw  [fill={rgb, 255:red, 197; green, 220; blue, 245 }  ,fill opacity=1 ] (341,177.07) .. controls (341,171.51) and (345.51,167) .. (351.07,167) -- (521.43,167) .. controls (526.99,167) and (531.5,171.51) .. (531.5,177.07) -- (531.5,207.27) .. controls (531.5,212.83) and (526.99,217.33) .. (521.43,217.33) -- (351.07,217.33) .. controls (345.51,217.33) and (341,212.83) .. (341,207.27) -- cycle ;

%Straight Lines [id:da38453818195490386] 
\draw    (389.5,45.33) -- (266.36,80.45) ;
\draw [shift={(264.43,81)}, rotate = 344.08000000000004] [color={rgb, 255:red, 0; green, 0; blue, 0 }  ][line width=0.75]    (10.93,-3.29) .. controls (6.95,-1.4) and (3.31,-0.3) .. (0,0) .. controls (3.31,0.3) and (6.95,1.4) .. (10.93,3.29)   ;
%Straight Lines [id:da4982354601391359] 
\draw    (226.17,44.33) -- (226.07,81) ;
\draw [shift={(226.07,83)}, rotate = 270.15] [color={rgb, 255:red, 0; green, 0; blue, 0 }  ][line width=0.75]    (10.93,-3.29) .. controls (6.95,-1.4) and (3.31,-0.3) .. (0,0) .. controls (3.31,0.3) and (6.95,1.4) .. (10.93,3.29)   ;
%Straight Lines [id:da033694103604325454] 
\draw    (227.5,133.33) -- (227.5,165.33) ;
\draw [shift={(227.5,167.33)}, rotate = 270] [color={rgb, 255:red, 0; green, 0; blue, 0 }  ][line width=0.75]    (10.93,-3.29) .. controls (6.95,-1.4) and (3.31,-0.3) .. (0,0) .. controls (3.31,0.3) and (6.95,1.4) .. (10.93,3.29)   ;
%Curve Lines [id:da9474003758420912] 
\draw    (263.5,28.33) .. controls (342.7,61) and (368.97,103.47) .. (390.84,162.54) ;
\draw [shift={(391.5,164.33)}, rotate = 249.86] [color={rgb, 255:red, 0; green, 0; blue, 0 }  ][line width=0.75]    (10.93,-3.29) .. controls (6.95,-1.4) and (3.31,-0.3) .. (0,0) .. controls (3.31,0.3) and (6.95,1.4) .. (10.93,3.29)   ;
%Straight Lines [id:da15716251072684195] 
\draw    (435.5,46.33) -- (437.47,163.33) ;
\draw [shift={(437.5,165.33)}, rotate = 269.04] [color={rgb, 255:red, 0; green, 0; blue, 0 }  ][line width=0.75]    (10.93,-3.29) .. controls (6.95,-1.4) and (3.31,-0.3) .. (0,0) .. controls (3.31,0.3) and (6.95,1.4) .. (10.93,3.29)   ;
%Straight Lines [id:da2978780738595006] 
\draw    (227,218) -- (227.47,251.33) ;
\draw [shift={(227.5,253.33)}, rotate = 269.19] [color={rgb, 255:red, 0; green, 0; blue, 0 }  ][line width=0.75]    (10.93,-3.29) .. controls (6.95,-1.4) and (3.31,-0.3) .. (0,0) .. controls (3.31,0.3) and (6.95,1.4) .. (10.93,3.29)   ;
%Straight Lines [id:da09380009558912938] 
\draw    (230.17,303.33) -- (278.89,339.15) ;
\draw [shift={(280.5,340.33)}, rotate = 216.32] [color={rgb, 255:red, 0; green, 0; blue, 0 }  ][line width=0.75]    (10.93,-3.29) .. controls (6.95,-1.4) and (3.31,-0.3) .. (0,0) .. controls (3.31,0.3) and (6.95,1.4) .. (10.93,3.29)   ;
%Straight Lines [id:da7487399163991604] 
\draw    (429.5,218.33) -- (326.79,339.81) ;
\draw [shift={(325.5,341.33)}, rotate = 310.22] [color={rgb, 255:red, 0; green, 0; blue, 0 }  ][line width=0.75]    (10.93,-3.29) .. controls (6.95,-1.4) and (3.31,-0.3) .. (0,0) .. controls (3.31,0.3) and (6.95,1.4) .. (10.93,3.29)   ;
%Rounded Rect [id:dp5677833542204889] 
\draw  [fill={rgb, 255:red, 197; green, 220; blue, 245 }  ,fill opacity=1 ] (199.78,352.07) .. controls (199.78,346.51) and (204.29,342) .. (209.85,342) -- (385.43,342) .. controls (390.99,342) and (395.5,346.51) .. (395.5,352.07) -- (395.5,382.27) .. controls (395.5,387.83) and (390.99,392.33) .. (385.43,392.33) -- (209.85,392.33) .. controls (204.29,392.33) and (199.78,387.83) .. (199.78,382.27) -- cycle ;

%Straight Lines [id:da005521743505217058] 
\draw    (302,392) -- (302.16,420.33) ;
\draw [shift={(302.17,422.33)}, rotate = 269.69] [color={rgb, 255:red, 0; green, 0; blue, 0 }  ][line width=0.75]    (10.93,-3.29) .. controls (6.95,-1.4) and (3.31,-0.3) .. (0,0) .. controls (3.31,0.3) and (6.95,1.4) .. (10.93,3.29)   ;

% Text Node
\draw (225.25,278.17) node   [align=left] {\begin{center}
Профиль добычи
\end{center}
};
% Text Node
\draw (435.25,191.17) node   [align=left] {\begin{center}
Модель обустройства
\end{center}
};
% Text Node
\draw (300.25,448.17) node   [align=left] {\begin{center}
Профиль NPV
\end{center}
};
% Text Node
\draw (228,25) node   [align=left] {PVT};
% Text Node
\draw (418,26) node   [align=left] {Многофазный поток};
% Text Node
\draw (228.25,108.17) node   [align=left] {\begin{center}
Модель скважины
\end{center}
};
% Text Node
\draw (227.25,192.17) node   [align=left] {\begin{center}
Модель пласта
\end{center}
};
% Text Node
\draw (296.61,366.17) node   [align=left] {\begin{center}
Финансово-экономическая\\ модель
\end{center}
};


\end{tikzpicture}
% font size needs to be fixed

    
\end{frame}